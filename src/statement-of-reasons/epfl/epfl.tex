\documentclass[11pt, a4paper]{article}

% packages ====

% custom margins.
\usepackage[a4paper,margin=1in]{geometry}
% create space between paragraphs instead of indentation.
\usepackage{parskip}
% custom headers and footers (e.g., page numbers).
\usepackage{fancyhdr}

% setup ====

\pagestyle{fancy}
% clear any default header and footer content.
\fancyhf{}
% set the center of the footer to display the page number.
\cfoot{\thepage}
% prevent a horizontal rule from appearing at the top of the page.
\renewcommand{\headrulewidth}{0pt}

% document ====

\begin{document}

\begin{center}
  {\LARGE Statement of Reasons} \\[.1cm]
  {\large MSc Computer Science at EPFL} \\[.9cm]
\end{center}

My goal is to build machine-learning systems that remain reliable when deployed in complex, real-world environments, especially in dynamic settings where data, physics and operational constraints must be handled together. I am applying to the MSc in Computer Science at EPFL because it offers an unusually strong combination of rigorous foundations (applied mathematics, learning theory and systems) and a research environment where I can turn scientific-ML ideas into tools that are both principled and usable.

I developed this direction by working at the intersection of research and production engineering throughout my Bachelor’s degree. Alongside my studies, I have spent three years as a part-time Software Engineer at Danfoss on a Kubernetes-hosted IoT platform with microservices and micro-frontends (30+ services, 400+ users and 3M+ real and simulated devices). This environment trained me to think in terms of distributed performance, consistency, operability and strengthened the way I collaborate. I learned how to communicate clearly and contribute in ways that help the whole team move faster.

I contributed to building a new micro-frontend for production and designed scalable UI patterns, including a feature-heavy entity tree with smart routing and cached results. I then intentionally moved to backend work to deepen my systems expertise. I quickly learned the stack and the languages used and delivered production contributions within my first week. Since then, I have owned a microservice for graph representations end-to-end, from requirements research and data-model design to API implementation and a sync engine that keeps graph and business entities consistent with the database. These experiences reflect how I work: I proactively network and communicate with people across the organization to build trust and create opportunities to take on more responsibilities, and I take ownership of turning proposals into production results. I intend to bring this combination of rapid learning and ownership to EPFL projects and research as well.

In parallel, my academic work moved me toward scientific machine learning. During my BSc at Ca’ Foscari University of Venice (completed with excellent results while working), I built a strong foundation in algorithms, operating systems and networking, and I particularly enjoyed mathematically grounded courses such as Numerical Algorithms, where computational techniques meet differential equations and modeling. This interest culminated in a research thesis on Physics-Informed Neural Networks (PINNs) for epidemiological compartmental models. In that work, I designed a systematic evaluation workflow, conducted ablation studies on key hyperparameters and proposed early-stopping criteria to improve training efficiency. Building on this, I am currently developing an open-source Python library for PINNs with an emphasis on modular architecture and developer experience, and I have already achieved substantial reductions in training time through workflow and stopping improvements.

More recently, I expanded my exposure to applied ML in production by contributing to Python-based data-processing pipelines (MageAI), interfacing with a data lake and training models for predictive control of energy consumption across business entities. This reinforced a core conviction: strong modeling ideas only become impactful when paired with robust engineering and careful reasoning about generalization, stability and causality.

EPFL is the right environment for me because it is one of the few places where I can deepen all the pillars I need (distributed systems engineering, rigorous machine learning and mathematically grounded modeling of dynamical systems) within a single MSc program, while being in close contact with an active research community. I am particularly interested in advanced coursework and thesis opportunities that connect learning methods with dynamical systems and robust inference, while also strengthening my systems background so I can build reproducible, deployable research software.

Finally, I am comfortable in international settings. During my Erasmus exchange at the University of Gothenburg, I complemented my curriculum with courses in machine learning and functional programming, and I hold a C2 English certification. I am motivated by challenging environments and I know how to sustain effort over time. Balancing a demanding part-time engineering role with a rigorous academic program has been my baseline for the past few years.

At EPFL, I aim to contribute as a student who not only performs academically but also builds reliable prototypes, clean research code and open-source tools that help ideas travel from papers to practice. I see the MSc as the step that will let me combine my experience with large-scale distributed systems and my growing focus on scientific ML into a coherent research and engineering profile.

\end{document}